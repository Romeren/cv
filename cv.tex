%%%%%%%%%%%%%%%%%
% This is an sample CV template created using altacv.cls
% (v1.1.3, 30 April 2017) written by LianTze Lim (liantze@gmail.com). Now compiles with pdfLaTeX, XeLaTeX and LuaLaTeX.
% 
%% It may be distributed and/or modified under the
%% conditions of the LaTeX Project Public License, either version 1.3
%% of this license or (at your option) any later version.
%% The latest version of this license is in
%%    http://www.latex-project.org/lppl.txt
%% and version 1.3 or later is part of all distributions of LaTeX
%% version 2003/12/01 or later.
%%%%%%%%%%%%%%%%

%% If you need to pass whatever options to xcolor
\PassOptionsToPackage{dvipsnames}{xcolor}

%% If you are using \orcid or academicons
%% icons, make sure you have the academicons 
%% option here, and compile with XeLaTeX
%% or LuaLaTeX.
% \documentclass[10pt,a4paper,academicons]{altacv}

%% Use the "normalphoto" option if you want a normal photo instead of cropped to a circle
% \documentclass[10pt,a4paper,normalphoto]{altacv}

\documentclass[10pt,a4paper]{altacv}

%% AltaCV uses the fontawesome and academicon fonts
%% and packages. 
%% See texdoc.net/pkg/fontawecome and http://texdoc.net/pkg/academicons for full list of symbols.
%% 
%% Compile with LuaLaTeX for best results. If you
%% want to use XeLaTeX, you may need to install
%% Academicons.ttf in your operating system's font 
%% folder.


% Change the page layout if you need to
\geometry{left=1cm,right=9cm,marginparwidth=6.8cm,marginparsep=1.2cm,top=0.5cm,bottom=1.25cm,footskip=2\baselineskip}

% Change the font if you want to.

% If using pdflatex:
\usepackage[utf8]{inputenc}
\usepackage[T1]{fontenc}
\usepackage[default]{lato}
\usepackage{hyperref}

% If using xelatex or lualatex:
% \setmainfont{Lato}

% Change the colours if you want to
\definecolor{Mulberry}{HTML}{031533}
\definecolor{SlateGrey}{HTML}{2E2E2E}
\definecolor{LightGrey}{HTML}{666666}
\definecolor{headingBlue}{HTML}{18499b}
\colorlet{heading}{headingBlue}
\colorlet{accent}{Mulberry}
\colorlet{emphasis}{SlateGrey}
\colorlet{body}{LightGrey}

% Change the bullets for itemize and rating marker
% for \cvskill if you want to
\renewcommand{\itemmarker}{{\small\textbullet}}
\renewcommand{\ratingmarker}{\faCircle}

%% sample.bib contains your publications
%\addbibresource{sample.bib}

\begin{document}
\name{Emil Sebastian Rømer}
\tagline{Software Engineer at Ramboll Denmark \& M.Sc. Software Engineering }
\photo{2.8cm}{images/migsdu}
\personalinfo{%
  % Not all of these are required!
  % You can add your own with \printinfo{symbol}{detail}
  \email{emilromer@hotmail.com}
  \phone{+45 3024 5719}
  \mailaddress{Vesterbro 59, st, 5000 Odense C}
  \location{Denmark}
%  \homepage{www.linkedin.com/in/romeren/}
%  \twitter{@twitterhandle}
  \linkedin{https://www.linkedin.com/in/romeren/}
%  \github{github.com/yourid}
  %% You MUST add the academicons option to \documentclass, then compile with LuaLaTeX or XeLaTeX, if you want to use \orcid or other academicons commands.
%   \orcid{orcid.org/0000-0000-0000-0000}
}

%% Make the header extend all the way to the right, if you want. 
\begin{fullwidth}
\makecvheader
\end{fullwidth}

%% Provide the file name containing the sidebar contents as an optional parameter to \cvsection.
%% You can always just use \marginpar{...} if you do
%% not need to align the top of the contents to any
%% \cvsection title in the "main" bar.

\marginpar{\vspace*{\dimexpr13pt-\baselineskip}\raggedright\cvsection{Projects}
%\cvachievement{\faCode}{Domain specific languages for building dashboard wep applications}{In order to deal with the high requirements of customizable dashboard applications for data exploration and visualization, a fellow student and I developed a prototype DSL for building custom dashboard applications}
 

\cvachievement{\faLeaf}{Image Recognition of Whales}{Together in a team, we started building a system that would recognize individual North Atlantic Right Whales from images, to help a group of marine biologists with tracking the species.}


\cvachievement{\faWifi}{Wifi Fingerprinting}{Through the use of WiFi sniffing and statistical machine learning, in collaboration with a team, we dynamically built models of buildings and estimated the location of the sniffing phone.}


%\cvachievement{\faPaintBrush}{Augmented spray can}{By augmenting the physical surroundings through a phone, together in a team, we turned a smart phone into a spray can, where everyone could \emph{tag} everything everywhere}

%\cvachievement{\faHeartbeat}{Pill Dispencer}{Together with an interdisciplinary team, we developed a prototype of a pill dispensing box that would keep track of the medicine in the box as well as dispense the right medicine at the right time}

\cvachievement{\faPlug}{Motion sensor}{In a 4-month project, a team and I developed a sensor that registered people walking on a stair. Through the use of an Arduino, data would be collected on a server.}

%\cvachievement{\faHeartbeat}{Common Diabetic}{Is a project that sat out to help people with diabetes through a Android app that helps the individual keeping track of medication}

%\cvachievement{\faChild}{Platform for teaching children}{In collaboration with the school of Søhus, a team and I developed and tested a prototype platform that can be used in schools to teach children}}

\marginpar{\vspace*{\dimexpr13pt-\baselineskip}\raggedright\cvsection{Programming Languages}

\cvskill{Python}{5}
\cvskill{Java}{4}
\cvskill{C\#}{4}
\cvskill{R}{4}
\cvskill{C++}{1}
\cvskill{VB}{1}}

%\marginpar{\vspace*{\dimexpr13pt-\baselineskip}\raggedright\cvsection{Languages}
\cvskill{Danish}{5}
\cvskill{English}{5}}

\marginpar{\vspace*{\dimexpr13pt-\baselineskip}\raggedright\cvsection{Skills}

%\cvskill{IFC2x3}{5}
\cvskill{IFC4 ADD2}{5}
%\cvskill{ISO 16739}{5}
\cvskill{.DWG}{4}
%\cvskill{sMAP}{4}
%\cvskill{HTML 5}{5}
%\cvskill{JQuery}{5}
%\cvskill{React}{2}
%\cvskill{Bootstrap}{5}
%\cvskill{Angular.JS}{1}

\cvskill{Web Development}{5}
\cvskill{DSL}{5}
%\cvskill{Interaction Design}{4}
\cvskill{Component-Based sw}{4}
%\cvskill{OSGi}{4}
%\cvskill{Swing}{4}
\cvskill{Linux}{4}
%\cvskill{Bash}{3}


\cvskill{Artificial Intelligence}{5}
\cvskill{Machine Learning}{4}
\cvskill{Simulation Modelling}{4}
\cvskill{Distributed Computing}{4}

%\cvskill{Amazon Aws}{4}
%\cvskill{MS Azure}{5}
%\cvskill{IBM Bluemix}{1}
\cvskill{Embedded}{3}
\cvskill{IoT}{4}
\cvskill{NoSQL}{4}
\cvskill{NewSql}{5}

}

\marginpar{\vspace*{\dimexpr13pt-\baselineskip}\raggedright\cvsection{Recognitions \& Publications}


\cvachievement{\faTrophy}{\href{http://www.sdu.dk/da/om_sdu/institutter_centre/centreforenergyinformatics/nyheder/2016+maj+emil+og+almir}{Danfose Engineering Tomorrow 2016}}{winner, with bachelor thesis on energy consumption}

\cvachievement{\faBook}{Publication}{ \href{http://findresearcher.sdu.dk/portal/da/publications/clustering-and-visualisation-of-electricity-data-to-identify-demand-response-opportunities(b58d7cf1-07d9-4125-98f1-5bee473b04b2).html}{Clustering and Visualisation of Electricity Data to identify Demand Response Opportunities: Poster Abstract} in collaboration with M.Sc Almir Mehanovic, Ph.D Jakob Hviid and Ph.D Mikkel Baun Kjærgaard}



}


\cvsection{Experience}
\cvevent{Software Engineer}{Ramboll Denmark}{06-2017 -- Ongoing}{Copenhagen}
\begin{itemize}
\item Designing simulation software for Facilities Management.
\end{itemize}

\divider


\cvevent{Student Programmer}{SDU, Mærsk McKinney Møller Institute}{09 2015 -- 06-2017}{Odense M}
\begin{itemize}
	\item Building low-energy portable Bluetooth trackers for occupancy detection.
\end{itemize}

\divider



\cvevent{Intern}{Ramboll Denmark}{08-2016 -- 12-2016 }{Copenhagen}
\begin{itemize}
	\item Researching simulation models for building decay. 
\end{itemize}

\divider


\cvevent{Student Programmer}{Powel Denmark}{12-2013 -- 08-2015}{Kolding}
\begin{itemize}
	\item Developing systems dealing with nationwide utilities infrastructure. 
\end{itemize}

\divider

\cvevent{3'rd Level IT Support (Volunteering)}{Roskilde Festival}{06-2013 -- 07-2017}{Roskilde}
\begin{itemize}
	\item Setup \& IT support of festival infrastructure.
\end{itemize}


\cvsection{Education}

\cvevent{M.Sc in Software Engineering}{University of Southern Denmark}{09-2014 -- 06-2017}{}
Thesis title: \emph{Decision support systems for budget optimization of building management}

In collaboration with Ramboll Denmark, The Region Capital Region of Copenhagen \& The Municipality of Hillerød

\divider

\cvevent{B.Sc in Software Engineering}{University of Southern Denmark}{09-2012 -- 06-2015}{}
Bachelor thesis title: \emph{Recognizing and visualizing energy consumption patterns of buildings using data mining}
In collaboration with the municipality of Odense
\divider

\cvsection{Languages}
\cvskill{Danish}{5}
\cvskill{English}{5}

\clearpage









\end{document}
