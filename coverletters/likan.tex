%%%%%%%%%%%%%%%%%
% This is an sample CV template created using altacv.cls
% (v1.1.3, 30 April 2017) written by LianTze Lim (liantze@gmail.com). Now compiles with pdfLaTeX, XeLaTeX and LuaLaTeX.
% 
%% It may be distributed and/or modified under the
%% conditions of the LaTeX Project Public License, either version 1.3
%% of this license or (at your option) any later version.
%% The latest version of this license is in
%%    http://www.latex-project.org/lppl.txt
%% and version 1.3 or later is part of all distributions of LaTeX
%% version 2003/12/01 or later.
%%%%%%%%%%%%%%%%

%% If you need to pass whatever options to xcolor
\PassOptionsToPackage{dvipsnames}{xcolor}

%% If you are using \orcid or academicons
%% icons, make sure you have the academicons 
%% option here, and compile with XeLaTeX
%% or LuaLaTeX.
% \documentclass[10pt,a4paper,academicons]{altacv}

%% Use the "normalphoto" option if you want a normal photo instead of cropped to a circle
% \documentclass[10pt,a4paper,normalphoto]{altacv}

\documentclass[12pt,a4paper]{altacv}

%% AltaCV uses the fontawesome and academicon fonts
%% and packages. 
%% See texdoc.net/pkg/fontawecome and http://texdoc.net/pkg/academicons for full list of symbols.
%% 
%% Compile with LuaLaTeX for best results. If you
%% want to use XeLaTeX, you may need to install
%% Academicons.ttf in your operating system's font 
%% folder.


% Change the page layout if you need to
\geometry{left=1cm,right=2cm,marginparwidth=6.8cm,marginparsep=1.2cm,top=0.5cm,bottom=1.25cm,footskip=2\baselineskip}

% Change the font if you want to.

% If using pdflatex:
\usepackage[utf8]{inputenc}
\usepackage[T1]{fontenc}
\usepackage[default]{lato}
\usepackage{hyperref}
\usepackage{setspace}

% If using xelatex or lualatex:
% \setmainfont{Lato}

% Change the colours if you want to
\definecolor{Mulberry}{HTML}{031533}
\definecolor{SlateGrey}{HTML}{2E2E2E}
\definecolor{LightGrey}{HTML}{666666}
\definecolor{headingBlue}{HTML}{18499b}
\colorlet{heading}{headingBlue}
\colorlet{accent}{Mulberry}
\colorlet{emphasis}{SlateGrey}
\colorlet{body}{SlateGrey}

% Change the bullets for itemize and rating marker
% for \cvskill if you want to
\renewcommand{\itemmarker}{{\small\textbullet}}
\renewcommand{\ratingmarker}{\faCircle}

\renewcommand{\baselinestretch}{1.5} 

%% sample.bib contains your publications
%\addbibresource{sample.bib}

\begin{document}
\name{Emil Sebastian Rømer}
\tagline{Software Engineer at Ramboll Denmark \& M.Sc. Software Engineering }
%\photo{2.8cm}{images/migsdu}
\personalinfo{%
  % Not all of these are required!
  % You can add your own with \printinfo{symbol}{detail}
  \email{emilromer@hotmail.com}
  \phone{+45 3024 5719}
  \mailaddress{Vesterbro 59, st, 5000 Odense C}
  \location{Denmark}
%  \homepage{www.linkedin.com/in/romeren/}
%  \twitter{@twitterhandle}
  \linkedin{https://www.linkedin.com/in/romeren/}
%  \github{github.com/yourid}
  %% You MUST add the academicons option to \documentclass, then compile with LuaLaTeX or XeLaTeX, if you want to use \orcid or other academicons commands.
%   \orcid{orcid.org/0000-0000-0000-0000}
}

%% Make the header extend all the way to the right, if you want. 

\makecvheader

%% Provide the file name containing the sidebar contents as an optional parameter to \cvsection.
%% You can always just use \marginpar{...} if you do
%% not need to align the top of the contents to any
%% \cvsection title in the "main" bar.


\cvsection{Ansøgning}
Hej Asmus Larsen \& Thomas Holm.
\\
\vspace{5mm}
Jeg stiftede første gang bekendtskab med Likan i forbindelse med mit kandidat speciale omhandlende Building Informatics, og fandt herefter hurtigt ud af at der er et overlap mellem det som jeg skrev mit speciale om og det som i laver i jeres virksomhed. 
På baggrund af dette fanger jeres virksomhed naturligt min interesse.
Jeg mener, at jeg kan tilføje jeres virksomhed værdig og derfor sender jeg jer denne uopfordret ansøgning.
\newline
Mit speciale havde som hovedfokus, at automatisere den strategiske planlægningen af vedligeholdelses procedure (forbyggende, genoprettende og akut), samt at prissætte disse aktiviteter i forhold til materiale og omfang af vedligeholdelses aktivitet.
Med min baggrund i Software Engineering udviklede jeg et simulerings værktøj der estimerer hvornår individuelle bygningsdele vil bryde ned og på baggrund af prisdata og AI planlægge den mest optimale strategi for vedligehold.
\vspace{5mm}
Jeg kan forstå at Likan, ligeledes udvikler denne type løsninger, dog i forhold til byggeriet og ikke selve driften af bygningen.
\\
Jeg vil derfor kunne trække på, både min brede viden indenfor IT og den dybdegående viden indenfor building informatics.
I forbindelse med min kandidat stiftede jeg bekendtskab med en række datamodeller for bygningsdata heriblandt den IFC standarden der er obligatorisk for alle byggeprojekter og essentiel for ethvert BIM system.
For uden mit speciale, har jeg de sidste to år arbejdet ved Mærsk McKinney Møller instituttets center of Energy Informatics. 
Her har jeg arbejdet tæt sammen med en række Ph.D studerende og professorer på udviklingen af projekter der alle kredser omkring BIMs.
Som software ingeniør har jeg kompetencerne til at designe og udvikle højkvalitets software løsninger der skaber værdig i praksis.
Jeg har erfaring med en bred vifte af programmeringssprog samt de nyeste teknologier indenfor Software Egineering og kunstig intelligens.

\vspace{5mm}
Jeg ser frem til at høre fra jer, på forhånd tak fordi i tog jer tid til at læse min ansøgning.
\\
\vspace{5mm}
Med Venlig hilsen,\\
\cvsubsection{Emil Sebastian Rømer}



\end{document}
