%%%%%%%%%%%%%%%%%
% This is an sample CV template created using altacv.cls
% (v1.1.3, 30 April 2017) written by LianTze Lim (liantze@gmail.com). Now compiles with pdfLaTeX, XeLaTeX and LuaLaTeX.
% 
%% It may be distributed and/or modified under the
%% conditions of the LaTeX Project Public License, either version 1.3
%% of this license or (at your option) any later version.
%% The latest version of this license is in
%%    http://www.latex-project.org/lppl.txt
%% and version 1.3 or later is part of all distributions of LaTeX
%% version 2003/12/01 or later.
%%%%%%%%%%%%%%%%

%% If you need to pass whatever options to xcolor
\PassOptionsToPackage{dvipsnames}{xcolor}

%% If you are using \orcid or academicons
%% icons, make sure you have the academicons 
%% option here, and compile with XeLaTeX
%% or LuaLaTeX.
% \documentclass[10pt,a4paper,academicons]{altacv}

%% Use the "normalphoto" option if you want a normal photo instead of cropped to a circle
% \documentclass[10pt,a4paper,normalphoto]{altacv}

\documentclass[12pt,a4paper]{altacv}

%% AltaCV uses the fontawesome and academicon fonts
%% and packages. 
%% See texdoc.net/pkg/fontawecome and http://texdoc.net/pkg/academicons for full list of symbols.
%% 
%% Compile with LuaLaTeX for best results. If you
%% want to use XeLaTeX, you may need to install
%% Academicons.ttf in your operating system's font 
%% folder.


% Change the page layout if you need to
\geometry{left=1cm,right=2cm,marginparwidth=6.8cm,marginparsep=1.2cm,top=0.5cm,bottom=1.25cm,footskip=2\baselineskip}

% Change the font if you want to.

% If using pdflatex:
\usepackage[utf8]{inputenc}
\usepackage[T1]{fontenc}
\usepackage[default]{lato}
\usepackage{hyperref}
\usepackage{setspace}

% If using xelatex or lualatex:
% \setmainfont{Lato}

% Change the colours if you want to
\definecolor{Mulberry}{HTML}{031533}
\definecolor{SlateGrey}{HTML}{2E2E2E}
\definecolor{LightGrey}{HTML}{666666}
\definecolor{headingBlue}{HTML}{18499b}
\colorlet{heading}{headingBlue}
\colorlet{accent}{Mulberry}
\colorlet{emphasis}{SlateGrey}
\colorlet{body}{SlateGrey}

% Change the bullets for itemize and rating marker
% for \cvskill if you want to
\renewcommand{\itemmarker}{{\small\textbullet}}
\renewcommand{\ratingmarker}{\faCircle}

\renewcommand{\baselinestretch}{1.5} 

%% sample.bib contains your publications
%\addbibresource{sample.bib}

\begin{document}
\name{Emil Sebastian Rømer}
\tagline{Software Engineer at Ramboll Denmark \& M.Sc. Software Engineering }
%\photo{2.8cm}{images/migsdu}
\personalinfo{%
  % Not all of these are required!
  % You can add your own with \printinfo{symbol}{detail}
  \email{emilromer@hotmail.com}
  \phone{+45 3024 5719}
  \mailaddress{Vesterbro 59, st, 5000 Odense C}
  \location{Denmark}
%  \homepage{www.linkedin.com/in/romeren/}
%  \twitter{@twitterhandle}
  \linkedin{https://www.linkedin.com/in/romeren/}
%  \github{github.com/yourid}
  %% You MUST add the academicons option to \documentclass, then compile with LuaLaTeX or XeLaTeX, if you want to use \orcid or other academicons commands.
%   \orcid{orcid.org/0000-0000-0000-0000}
}

%% Make the header extend all the way to the right, if you want. 

\makecvheader

%% Provide the file name containing the sidebar contents as an optional parameter to \cvsection.
%% You can always just use \marginpar{...} if you do
%% not need to align the top of the contents to any
%% \cvsection title in the "main" bar.


\cvsection{Application: SW Developer}
Dear Danfoss.
\\
\vspace{5mm}

It is with great excitement that I submit my application to Danfoss's SW developer position.
\newline

I am a graduate within Software Engineering and with great interest in data analytics, IoT and Energy Informatics.
\\
With my bachelor in Software engineering I have knowledge within all development phases from idea to product.
I know the consequences that an early decision might have on later development, and therefore understand the importance of architectural design as well as the importance of building a system that is both customizable and maintainable.
In my Master's in Software engineering, I studied and applied advanced system designs to overcome many of the problems with the current market.
I know that an IT business cannot exists if it is too slow in bringing its products to the market, or by delivering the wrong or faulty products.
In order to deal with these risks, I have worked with both systems and processes designed to continuously deliver and deploy incremental solutions.
During my studies I have also been acquainted with many different agile processes including SCRUM which i have also used during my work at Powel Denmark.



\vspace{5mm}
Thank you in advance for the time you dedicated to my application. I am looking forward to hearing from you.
\\
\vspace{5mm}
Best regards,\\
\cvsubsection{Emil Sebastian Rømer}



\end{document}
