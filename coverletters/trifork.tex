%%%%%%%%%%%%%%%%%
% This is an sample CV template created using altacv.cls
% (v1.1.3, 30 April 2017) written by LianTze Lim (liantze@gmail.com). Now compiles with pdfLaTeX, XeLaTeX and LuaLaTeX.
% 
%% It may be distributed and/or modified under the
%% conditions of the LaTeX Project Public License, either version 1.3
%% of this license or (at your option) any later version.
%% The latest version of this license is in
%%    http://www.latex-project.org/lppl.txt
%% and version 1.3 or later is part of all distributions of LaTeX
%% version 2003/12/01 or later.
%%%%%%%%%%%%%%%%

%% If you need to pass whatever options to xcolor
\PassOptionsToPackage{dvipsnames}{xcolor}

%% If you are using \orcid or academicons
%% icons, make sure you have the academicons 
%% option here, and compile with XeLaTeX
%% or LuaLaTeX.
% \documentclass[10pt,a4paper,academicons]{altacv}

%% Use the "normalphoto" option if you want a normal photo instead of cropped to a circle
% \documentclass[10pt,a4paper,normalphoto]{altacv}

\documentclass[10pt,a4paper]{altacv}

%% AltaCV uses the fontawesome and academicon fonts
%% and packages. 
%% See texdoc.net/pkg/fontawecome and http://texdoc.net/pkg/academicons for full list of symbols.
%% 
%% Compile with LuaLaTeX for best results. If you
%% want to use XeLaTeX, you may need to install
%% Academicons.ttf in your operating system's font 
%% folder.


% Change the page layout if you need to
\geometry{left=1cm,right=2cm,marginparwidth=6.8cm,marginparsep=1.2cm,top=0.5cm,bottom=1.25cm,footskip=2\baselineskip}

% Change the font if you want to.

% If using pdflatex:
\usepackage[utf8]{inputenc}
\usepackage[T1]{fontenc}
\usepackage[default]{lato}
\usepackage{hyperref}
\usepackage{setspace}

% If using xelatex or lualatex:
% \setmainfont{Lato}

% Change the colours if you want to
\definecolor{Mulberry}{HTML}{031533}
\definecolor{SlateGrey}{HTML}{2E2E2E}
\definecolor{LightGrey}{HTML}{666666}
\definecolor{headingBlue}{HTML}{18499b}
\colorlet{heading}{headingBlue}
\colorlet{accent}{Mulberry}
\colorlet{emphasis}{SlateGrey}
\colorlet{body}{SlateGrey}

% Change the bullets for itemize and rating marker
% for \cvskill if you want to
\renewcommand{\itemmarker}{{\small\textbullet}}
\renewcommand{\ratingmarker}{\faCircle}

\renewcommand{\baselinestretch}{1.5} 

%% sample.bib contains your publications
%\addbibresource{sample.bib}

\begin{document}
\name{Emil Sebastian Rømer}
\tagline{Software Engineer at Ramboll Denmark \& M.Sc. Software Engineering }
%\photo{2.8cm}{images/migsdu}
\personalinfo{%
  % Not all of these are required!
  % You can add your own with \printinfo{symbol}{detail}
  \email{emilromer@hotmail.com}
  \phone{+45 3024 5719}
  \mailaddress{Vesterbro 59, st, 5000 Odense C}
  \location{Denmark}
%  \homepage{www.linkedin.com/in/romeren/}
%  \twitter{@twitterhandle}
  \linkedin{https://www.linkedin.com/in/romeren/}
%  \github{github.com/yourid}
  %% You MUST add the academicons option to \documentclass, then compile with LuaLaTeX or XeLaTeX, if you want to use \orcid or other academicons commands.
%   \orcid{orcid.org/0000-0000-0000-0000}
}

%% Make the header extend all the way to the right, if you want. 

\makecvheader

%% Provide the file name containing the sidebar contents as an optional parameter to \cvsection.
%% You can always just use \marginpar{...} if you do
%% not need to align the top of the contents to any
%% \cvsection title in the "main" bar.



\cvsection{Ansøgning: Softwareudvikler}
Til Trifork.
\\
\vspace{5mm}
Hen over de sidste tre år har jeg deltaget i GOTO konferencerne som afholdtes i København, og derigennem har jeg både snakket med ansatte og hørt meget om Trifork som virksomhed.
Jeg ser nu Trifork som værende både en frontløber indenfor software udvikling og et kvalitets stemple, -og med denne baggrund, er det mig en fornøjelse at jeg hermed indsender min ansøgning til stillingen som softwareudvikler i København.
\\
\vspace{5mm}
Jeg er nyuddannet ingeniør med en kandidat i Software Engineering som, i modsætningen til en klassisk datalog uddannelse, har fokus på at skabe kvalitets software gennem arkitektur, design-patterns og processer.
Foruden dette, bærer har min uddannelse stærkt præg af at være løsnings-orienteret, med store projekter i samarbejde med virksomheder hvert semester.
Disse projekter er blevet bedømt efter deres effektivitet og kvalitet, og det er dette der har lært mig altid at stræbe efter den mest optimale løsning.
Derudover, her de regelmæssige projekter gjort mig god til at arbejde sammen med andre i teams, samt givet muligheden for at teste flere agile process-metoder af i praksis.
\\
\vspace{5mm}
Ved siden af mit studie har jeg de sidste to år arbejdet ved Ramboll og Mærsk McKinney Møller Instituttets (MMMI) center for Energy Informatics.
Ved Ramboll har jeg arbejdet tværfagligt i teams med andre ingeniører og på egen hånd som konsulent for Rambolls kunder.
Dette har gjort mig god til både at formilde tekniske detaljer til andre, samt at forstå kunders problematikker og finde den rigtige løsning herpå.
Ved MMMI har jeg arbejdet sammen med Ph.D studerende og professorer med det jeg allermest brænder for; cutting-edge teknologier til IoT og Building Informatics.
\\
\vspace{5mm} 
Jeg ser frem til at høre fra jer, på forhånd tak fordi i tog jer tid til at læse min ansøgning.


\vspace{5mm}
Med venlig hilsen,\\
\cvsubsection{Emil Sebastian Rømer}


\end{document}
